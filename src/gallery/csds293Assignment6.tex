%! Author = viniemm
%! Date = 10/6/21

\title{CSDS 293: Software Craftsmanship\newline Assignment 6}

\documentclass{article}
\usepackage{amsmath}
\usepackage{amssymb}
\usepackage{tikz}
\usepackage[utf8]{inputenc}
\usepackage[utf8]{inputenc}
\usepackage{graphicx}


\title{CSDS 293: Software Craftsmanship\newline Assignment 6}
\author{Vinayak Mathur vxm167}
\date{Submitted: October 7th 2021}

\begin{document}

\maketitle

\section*{Problem}

\section*{Solution}

\section{Problem 1}
$\Rightarrow$Construct a directed Graph G(V,E) where the Vertices (V) are each of the teams and the Edges (E) $uv$ correspond to the direction that u was beaten by v. Each node should also contain a field called "Z" which will be the domination factor of that team. $Z=\infty$ at $t=0$.\\
\\
$\Rightarrow$Run DFS on the graph starting from the best ranking team from last year. \\ \\
Modify the DFS such that vertices have a Z field that corresponds to their domination factor.\\ \\
$\Rightarrow$ Initialize all of the domination factors as $\infty$ $(v.z=\infty)$ for every vertex $v\in V$\\ \\
$\Rightarrow$ At each outer for loop create a root constant and prass it to DFS-visit. In the DFS-visit, if the root constant is smaller than the current domination factor, then update the domination factor.\\
Proof: We would like to prove that the procedure finds the correct domination factor.\\ \\
Assume $z(x)[domination factor]=r(v)[root]$ for some v\\
$\Rightarrow\ \exists$ a path from v to x in the graph and no other vertex on that path has a better ranking factor that v. \\ \\
case-1: Single path from v to x \\
$\raisebox{.5pt}{\textcircled{\raisebox{-.9pt} {v}}}\rightarrow \circ \rightarrow \circ \rightarrow \circ \rightarrow \raisebox{.5pt}{\textcircled{\raisebox{-.9pt} {u}}} \rightarrow \circ \rightarrow \raisebox{.5pt}{\textcircled{\raisebox{-.9pt} {x}}}$\\
$\Rightarrow$ if u has a better ranking, then x is going to get u's ranking.\\
$\Rightarrow$ Since we visit vertices in order of ranking this path must be white at time v.d \\
$\Rightarrow$ then from white path theorem, x is the descendant of v or the vertex just before the x needs to be the descendant of v. \newpage
\noindent case-2: multiple paths to x then v must be best ranking.\\
\indent \indent $\raisebox{.5pt}{\textcircled{\raisebox{-.9pt} {s}}}$\\
\indent \indent$\downarrow$ \\
$\raisebox{.5pt}{\textcircled{\raisebox{-.9pt} {v}}}\rightarrow \raisebox{.5pt}{\textcircled{\raisebox{-.9pt} {u}}} \rightarrow\circ \rightarrow \circ \rightarrow \circ \rightarrow  \raisebox{.5pt}{\textcircled{\raisebox{-.9pt} {u}}} \rightarrow \raisebox{.5pt}{\textcircled{\raisebox{-.9pt} {x}}}$\\
$\Rightarrow$ if there is a node s such that s has a better ranking than v, then node u will get the ranking from s.\\
$\Rightarrow$ This is a contradiction and hence s cannot exist.
\section{Problem 2}
a) $uv\in E$ if $v.f>u.f$ then $uv$ must be part of a cycle. \\
$\raisebox{.5pt}{\textcircled{\raisebox{-.9pt} {u}}} \rightarrow \raisebox{.5pt}{\textcircled{\raisebox{-.9pt} {v}}}$\\
$u.d<v.d$ v is white \\
b) Counter example: \\
$u\Rightarrow$ $u.d = 4$, $u.f = 5$\\
$v\Rightarrow$ $v.d = 2$, $v.f = 3$\\
$w\Rightarrow$ $w.d = 1$, $w.f = 6$\\
If there is an edge from u to v then it will be a cross edge.\\
\section{Problem 3}
.\hrulefill
\newline
.\hrulefill
\newpage
\end{document}
